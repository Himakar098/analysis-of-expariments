% Options for packages loaded elsewhere
\PassOptionsToPackage{unicode}{hyperref}
\PassOptionsToPackage{hyphens}{url}
%
\documentclass[
]{article}
\usepackage{amsmath,amssymb}
\usepackage{iftex}
\ifPDFTeX
  \usepackage[T1]{fontenc}
  \usepackage[utf8]{inputenc}
  \usepackage{textcomp} % provide euro and other symbols
\else % if luatex or xetex
  \usepackage{unicode-math} % this also loads fontspec
  \defaultfontfeatures{Scale=MatchLowercase}
  \defaultfontfeatures[\rmfamily]{Ligatures=TeX,Scale=1}
\fi
\usepackage{lmodern}
\ifPDFTeX\else
  % xetex/luatex font selection
\fi
% Use upquote if available, for straight quotes in verbatim environments
\IfFileExists{upquote.sty}{\usepackage{upquote}}{}
\IfFileExists{microtype.sty}{% use microtype if available
  \usepackage[]{microtype}
  \UseMicrotypeSet[protrusion]{basicmath} % disable protrusion for tt fonts
}{}
\makeatletter
\@ifundefined{KOMAClassName}{% if non-KOMA class
  \IfFileExists{parskip.sty}{%
    \usepackage{parskip}
  }{% else
    \setlength{\parindent}{0pt}
    \setlength{\parskip}{6pt plus 2pt minus 1pt}}
}{% if KOMA class
  \KOMAoptions{parskip=half}}
\makeatother
\usepackage{xcolor}
\usepackage[margin=1in]{geometry}
\usepackage{color}
\usepackage{fancyvrb}
\newcommand{\VerbBar}{|}
\newcommand{\VERB}{\Verb[commandchars=\\\{\}]}
\DefineVerbatimEnvironment{Highlighting}{Verbatim}{commandchars=\\\{\}}
% Add ',fontsize=\small' for more characters per line
\usepackage{framed}
\definecolor{shadecolor}{RGB}{248,248,248}
\newenvironment{Shaded}{\begin{snugshade}}{\end{snugshade}}
\newcommand{\AlertTok}[1]{\textcolor[rgb]{0.94,0.16,0.16}{#1}}
\newcommand{\AnnotationTok}[1]{\textcolor[rgb]{0.56,0.35,0.01}{\textbf{\textit{#1}}}}
\newcommand{\AttributeTok}[1]{\textcolor[rgb]{0.13,0.29,0.53}{#1}}
\newcommand{\BaseNTok}[1]{\textcolor[rgb]{0.00,0.00,0.81}{#1}}
\newcommand{\BuiltInTok}[1]{#1}
\newcommand{\CharTok}[1]{\textcolor[rgb]{0.31,0.60,0.02}{#1}}
\newcommand{\CommentTok}[1]{\textcolor[rgb]{0.56,0.35,0.01}{\textit{#1}}}
\newcommand{\CommentVarTok}[1]{\textcolor[rgb]{0.56,0.35,0.01}{\textbf{\textit{#1}}}}
\newcommand{\ConstantTok}[1]{\textcolor[rgb]{0.56,0.35,0.01}{#1}}
\newcommand{\ControlFlowTok}[1]{\textcolor[rgb]{0.13,0.29,0.53}{\textbf{#1}}}
\newcommand{\DataTypeTok}[1]{\textcolor[rgb]{0.13,0.29,0.53}{#1}}
\newcommand{\DecValTok}[1]{\textcolor[rgb]{0.00,0.00,0.81}{#1}}
\newcommand{\DocumentationTok}[1]{\textcolor[rgb]{0.56,0.35,0.01}{\textbf{\textit{#1}}}}
\newcommand{\ErrorTok}[1]{\textcolor[rgb]{0.64,0.00,0.00}{\textbf{#1}}}
\newcommand{\ExtensionTok}[1]{#1}
\newcommand{\FloatTok}[1]{\textcolor[rgb]{0.00,0.00,0.81}{#1}}
\newcommand{\FunctionTok}[1]{\textcolor[rgb]{0.13,0.29,0.53}{\textbf{#1}}}
\newcommand{\ImportTok}[1]{#1}
\newcommand{\InformationTok}[1]{\textcolor[rgb]{0.56,0.35,0.01}{\textbf{\textit{#1}}}}
\newcommand{\KeywordTok}[1]{\textcolor[rgb]{0.13,0.29,0.53}{\textbf{#1}}}
\newcommand{\NormalTok}[1]{#1}
\newcommand{\OperatorTok}[1]{\textcolor[rgb]{0.81,0.36,0.00}{\textbf{#1}}}
\newcommand{\OtherTok}[1]{\textcolor[rgb]{0.56,0.35,0.01}{#1}}
\newcommand{\PreprocessorTok}[1]{\textcolor[rgb]{0.56,0.35,0.01}{\textit{#1}}}
\newcommand{\RegionMarkerTok}[1]{#1}
\newcommand{\SpecialCharTok}[1]{\textcolor[rgb]{0.81,0.36,0.00}{\textbf{#1}}}
\newcommand{\SpecialStringTok}[1]{\textcolor[rgb]{0.31,0.60,0.02}{#1}}
\newcommand{\StringTok}[1]{\textcolor[rgb]{0.31,0.60,0.02}{#1}}
\newcommand{\VariableTok}[1]{\textcolor[rgb]{0.00,0.00,0.00}{#1}}
\newcommand{\VerbatimStringTok}[1]{\textcolor[rgb]{0.31,0.60,0.02}{#1}}
\newcommand{\WarningTok}[1]{\textcolor[rgb]{0.56,0.35,0.01}{\textbf{\textit{#1}}}}
\usepackage{graphicx}
\makeatletter
\def\maxwidth{\ifdim\Gin@nat@width>\linewidth\linewidth\else\Gin@nat@width\fi}
\def\maxheight{\ifdim\Gin@nat@height>\textheight\textheight\else\Gin@nat@height\fi}
\makeatother
% Scale images if necessary, so that they will not overflow the page
% margins by default, and it is still possible to overwrite the defaults
% using explicit options in \includegraphics[width, height, ...]{}
\setkeys{Gin}{width=\maxwidth,height=\maxheight,keepaspectratio}
% Set default figure placement to htbp
\makeatletter
\def\fps@figure{htbp}
\makeatother
\setlength{\emergencystretch}{3em} % prevent overfull lines
\providecommand{\tightlist}{%
  \setlength{\itemsep}{0pt}\setlength{\parskip}{0pt}}
\setcounter{secnumdepth}{-\maxdimen} % remove section numbering
\ifLuaTeX
  \usepackage{selnolig}  % disable illegal ligatures
\fi
\IfFileExists{bookmark.sty}{\usepackage{bookmark}}{\usepackage{hyperref}}
\IfFileExists{xurl.sty}{\usepackage{xurl}}{} % add URL line breaks if available
\urlstyle{same}
\hypersetup{
  pdftitle={AOE Assignment 2},
  pdfauthor={HimakarGadham \& AtikantJain},
  hidelinks,
  pdfcreator={LaTeX via pandoc}}

\title{AOE Assignment 2}
\author{HimakarGadham \& AtikantJain}
\date{2024-04-27}

\begin{document}
\maketitle

\hypertarget{assignment-2-submitted-by-student-name-student-id}{%
\subsection{Assignment 2 submitted by Student Name, Student
ID}\label{assignment-2-submitted-by-student-name-student-id}}

Himakar Gadham - 23783777, Atikant Jain - 24051868

\hypertarget{statement-of-contribution}{%
\subsubsection{Statement of
Contribution:}\label{statement-of-contribution}}

Both contributed equally to all the questions. We both sat together in
library and worked through each question. We did it by going through
labs question and answers, comparing our output with each other,
clarifying doubts and clearing concepts.

Question 1 Air Pollution and Mortality. Does pollution kill people? (30
marks) (a) (5 marks) Carry out exploratory data analysis (EDA) of this
dataset, taking into account the following groups of associations with
the response: ˆ 1 plot for Climate ˆ 1 plot for Socioeconomic ˆ 1 plot for
Pollution ˆ 1 plot combining the comparisons of the response against
State.Code and Region respectively. Hint. Your answer must include 4
plots for EDA including interpretation in a few concise sentences. Do
not include the R code.

Answer:

\begin{enumerate}
\def\labelenumi{\arabic{enumi}.}
\tightlist
\item
  \textbf{Climate Variables vs.~Mortality}:

  \begin{itemize}
  \tightlist
  \item
    This plot shows weak correlations between climate variables
    (Precipitation, Humidity, January Temperature, July Temperature) and
    mortality, suggesting that climate alone might not be a strong
    predictor of mortality rates.
  \end{itemize}
\item
  \textbf{Socioeconomic Variables vs.~Mortality}:

  \begin{itemize}
  \tightlist
  \item
    Socioeconomic factors including the percentage of the population
    over 65, population per household, median education years, and
    population density exhibit varying degrees of correlation with
    mortality rates. Variables like education levels and household
    population density could potentially influence mortality, indicating
    the need for further detailed regression analysis to understand
    these impacts better.
  \end{itemize}
\item
  \textbf{Pollution Variables vs.~Mortality}:

  \begin{itemize}
  \tightlist
  \item
    Pollution variables (Hydrocarbons, Oxides of Nitrogen, Sulphur
    Dioxide) show potential correlations where higher levels of
    pollutants might be associated with increased mortality rates. This
    suggests a direct inquiry into pollution effects is necessary,
    adjusting for other socioeconomic and climate variables.
  \end{itemize}
\item
  \textbf{Mortality by Region}:

  \begin{itemize}
  \tightlist
  \item
    Mortality rates appear to vary significantly across regions, which
    might reflect underlying differences in environmental policies,
    access to healthcare, lifestyle, or even unmeasured pollution
    levels.
  \end{itemize}
\end{enumerate}

These findings from the EDA are crucial as they guide the deeper
investigation into how pollution specifically impacts mortality,
controlling for socioeconomic and climate factors. Next, we can progress
into regression analysis to model these relationships more formally and
perform statistical tests to determine the significance and strength of
pollution's impact on mortality.

\begin{Shaded}
\begin{Highlighting}[]
\DocumentationTok{\#\# Step 1: Model Selection}
\FunctionTok{library}\NormalTok{(leaps)}
\FunctionTok{library}\NormalTok{(MASS)}

\CommentTok{\# Load your data}
\NormalTok{data }\OtherTok{\textless{}{-}} \FunctionTok{read.csv}\NormalTok{(}\StringTok{"Pollution.csv"}\NormalTok{)}

\CommentTok{\# All subsets regression with a limit on the number of variables}
\NormalTok{fit }\OtherTok{\textless{}{-}} \FunctionTok{regsubsets}\NormalTok{(Mortality }\SpecialCharTok{\textasciitilde{}}\NormalTok{ Over65 }\SpecialCharTok{+}\NormalTok{ House }\SpecialCharTok{+}\NormalTok{ Educ }\SpecialCharTok{+}\NormalTok{ Sound }\SpecialCharTok{+}\NormalTok{ Density }\SpecialCharTok{+}\NormalTok{ NonWhite }\SpecialCharTok{+}\NormalTok{ WhiteCol }\SpecialCharTok{+}\NormalTok{ Poor }\SpecialCharTok{+}\NormalTok{ Precip }\SpecialCharTok{+}\NormalTok{ Humidity }\SpecialCharTok{+}\NormalTok{ JanTemp }\SpecialCharTok{+}\NormalTok{ JulyTemp }\SpecialCharTok{+}\NormalTok{ HC }\SpecialCharTok{+}\NormalTok{ NOX }\SpecialCharTok{+}\NormalTok{ SO2, }\AttributeTok{data=}\NormalTok{data, }\AttributeTok{nbest=}\DecValTok{1}\NormalTok{, }\AttributeTok{nvmax=}\DecValTok{10}\NormalTok{, }\AttributeTok{method=}\StringTok{"exhaustive"}\NormalTok{)}

\CommentTok{\# View the summary of the best models}
\FunctionTok{summary}\NormalTok{(fit)}
\end{Highlighting}
\end{Shaded}

\begin{verbatim}
## Subset selection object
## Call: regsubsets.formula(Mortality ~ Over65 + House + Educ + Sound + 
##     Density + NonWhite + WhiteCol + Poor + Precip + Humidity + 
##     JanTemp + JulyTemp + HC + NOX + SO2, data = data, nbest = 1, 
##     nvmax = 10, method = "exhaustive")
## 15 Variables  (and intercept)
##          Forced in Forced out
## Over65       FALSE      FALSE
## House        FALSE      FALSE
## Educ         FALSE      FALSE
## Sound        FALSE      FALSE
## Density      FALSE      FALSE
## NonWhite     FALSE      FALSE
## WhiteCol     FALSE      FALSE
## Poor         FALSE      FALSE
## Precip       FALSE      FALSE
## Humidity     FALSE      FALSE
## JanTemp      FALSE      FALSE
## JulyTemp     FALSE      FALSE
## HC           FALSE      FALSE
## NOX          FALSE      FALSE
## SO2          FALSE      FALSE
## 1 subsets of each size up to 10
## Selection Algorithm: exhaustive
##           Over65 House Educ Sound Density NonWhite WhiteCol Poor Precip
## 1  ( 1 )  " "    " "   " "  " "   " "     "*"      " "      " "  " "   
## 2  ( 1 )  " "    " "   "*"  " "   " "     "*"      " "      " "  " "   
## 3  ( 1 )  " "    " "   "*"  " "   " "     "*"      " "      " "  " "   
## 4  ( 1 )  " "    " "   " "  " "   " "     "*"      " "      " "  "*"   
## 5  ( 1 )  " "    " "   "*"  " "   " "     "*"      " "      " "  "*"   
## 6  ( 1 )  " "    " "   "*"  " "   " "     "*"      " "      " "  "*"   
## 7  ( 1 )  " "    "*"   "*"  " "   " "     "*"      " "      " "  "*"   
## 8  ( 1 )  "*"    "*"   "*"  " "   " "     "*"      " "      " "  "*"   
## 9  ( 1 )  "*"    "*"   "*"  " "   " "     "*"      " "      " "  "*"   
## 10  ( 1 ) "*"    "*"   "*"  " "   "*"     "*"      " "      " "  "*"   
##           Humidity JanTemp JulyTemp HC  NOX SO2
## 1  ( 1 )  " "      " "     " "      " " " " " "
## 2  ( 1 )  " "      " "     " "      " " " " " "
## 3  ( 1 )  " "      "*"     " "      " " " " " "
## 4  ( 1 )  " "      "*"     " "      " " " " "*"
## 5  ( 1 )  " "      "*"     " "      " " " " "*"
## 6  ( 1 )  " "      "*"     "*"      " " " " "*"
## 7  ( 1 )  " "      "*"     "*"      " " " " "*"
## 8  ( 1 )  " "      "*"     "*"      " " " " "*"
## 9  ( 1 )  " "      "*"     "*"      "*" "*" " "
## 10  ( 1 ) " "      "*"     "*"      "*" "*" " "
\end{verbatim}

\begin{Shaded}
\begin{Highlighting}[]
\CommentTok{\# Plotting the models based on BIC, Adjusted R{-}squared, and Cp}
\FunctionTok{plot}\NormalTok{(fit, }\AttributeTok{scale=}\StringTok{"bic"}\NormalTok{)}
\end{Highlighting}
\end{Shaded}

\includegraphics{Gadham_23783777_Atikant_24051868_files/figure-latex/unnamed-chunk-1-1.pdf}

\begin{Shaded}
\begin{Highlighting}[]
\FunctionTok{plot}\NormalTok{(fit, }\AttributeTok{scale=}\StringTok{"adjr2"}\NormalTok{)}
\end{Highlighting}
\end{Shaded}

\includegraphics{Gadham_23783777_Atikant_24051868_files/figure-latex/unnamed-chunk-1-2.pdf}

\begin{Shaded}
\begin{Highlighting}[]
\FunctionTok{plot}\NormalTok{(fit, }\AttributeTok{scale=}\StringTok{"Cp"}\NormalTok{)}
\end{Highlighting}
\end{Shaded}

\includegraphics{Gadham_23783777_Atikant_24051868_files/figure-latex/unnamed-chunk-1-3.pdf}

\begin{Shaded}
\begin{Highlighting}[]
\DocumentationTok{\#\#\#\# Step 2: Fit the Optimal Model}
\CommentTok{\# Assuming the best model includes certain variables based on previous step}
\NormalTok{model }\OtherTok{\textless{}{-}} \FunctionTok{lm}\NormalTok{(Mortality }\SpecialCharTok{\textasciitilde{}}\NormalTok{ Over65 }\SpecialCharTok{+}\NormalTok{ Educ }\SpecialCharTok{+}\NormalTok{ HC }\SpecialCharTok{+}\NormalTok{ NOX }\SpecialCharTok{+}\NormalTok{ SO2, }\AttributeTok{data=}\NormalTok{data)}
\FunctionTok{summary}\NormalTok{(model)}
\end{Highlighting}
\end{Shaded}

\begin{verbatim}
## 
## Call:
## lm(formula = Mortality ~ Over65 + Educ + HC + NOX + SO2, data = data)
## 
## Residuals:
##      Min       1Q   Median       3Q      Max 
## -115.310  -25.516   -1.329   30.719  115.979 
## 
## Coefficients:
##               Estimate Std. Error t value Pr(>|t|)    
## (Intercept) 1330.30802  100.05968  13.295  < 2e-16 ***
## Over65       -11.41380    4.14927  -2.751  0.00807 ** 
## Educ         -27.60028    7.97921  -3.459  0.00107 ** 
## HC            -1.48302    0.54870  -2.703  0.00917 ** 
## NOX            2.85463    1.13746   2.510  0.01512 *  
## SO2            0.08951    0.15533   0.576  0.56681    
## ---
## Signif. codes:  0 '***' 0.001 '**' 0.01 '*' 0.05 '.' 0.1 ' ' 1
## 
## Residual standard error: 45.9 on 54 degrees of freedom
## Multiple R-squared:  0.5016, Adjusted R-squared:  0.4554 
## F-statistic: 10.87 on 5 and 54 DF,  p-value: 2.893e-07
\end{verbatim}

\begin{Shaded}
\begin{Highlighting}[]
\DocumentationTok{\#\#\#\# Step 3: Model Diagnostics}
\CommentTok{\# Checking for assumptions: linearity, normality, and homoscedasticity}
\FunctionTok{par}\NormalTok{(}\AttributeTok{mfrow=}\FunctionTok{c}\NormalTok{(}\DecValTok{2}\NormalTok{,}\DecValTok{2}\NormalTok{))}
\FunctionTok{plot}\NormalTok{(model)}
\end{Highlighting}
\end{Shaded}

\includegraphics{Gadham_23783777_Atikant_24051868_files/figure-latex/unnamed-chunk-3-1.pdf}

\begin{Shaded}
\begin{Highlighting}[]
\DocumentationTok{\#\#\#\# Step 4: Testing Pollution Variables\textquotesingle{} Impact}
\CommentTok{\# Add pollution variables to a base model and perform an F{-}test}
\NormalTok{base\_model }\OtherTok{\textless{}{-}} \FunctionTok{lm}\NormalTok{(Mortality }\SpecialCharTok{\textasciitilde{}}\NormalTok{ Over65 }\SpecialCharTok{+}\NormalTok{ Educ, }\AttributeTok{data=}\NormalTok{data)}
\NormalTok{pollution\_model }\OtherTok{\textless{}{-}} \FunctionTok{update}\NormalTok{(base\_model, .}\SpecialCharTok{\textasciitilde{}}\NormalTok{. }\SpecialCharTok{+} \FunctionTok{log}\NormalTok{(HC) }\SpecialCharTok{+} \FunctionTok{log}\NormalTok{(NOX) }\SpecialCharTok{+} \FunctionTok{log}\NormalTok{(SO2))}

\CommentTok{\# F{-}test}
\FunctionTok{anova}\NormalTok{(base\_model, pollution\_model)}
\end{Highlighting}
\end{Shaded}

\begin{verbatim}
## Analysis of Variance Table
## 
## Model 1: Mortality ~ Over65 + Educ
## Model 2: Mortality ~ Over65 + Educ + log(HC) + log(NOX) + log(SO2)
##   Res.Df    RSS Df Sum of Sq      F   Pr(>F)   
## 1     57 154638                                
## 2     54 118950  3     35689 5.4005 0.002528 **
## ---
## Signif. codes:  0 '***' 0.001 '**' 0.01 '*' 0.05 '.' 0.1 ' ' 1
\end{verbatim}

\begin{enumerate}
\def\labelenumi{(\alph{enumi})}
\setcounter{enumi}{1}
\tightlist
\item
  (9 marks) Perform model selection process using All Subset method to
  arrive at good regression models that account for variation in
  mortality between the cities that can be attributed to di erences in
  climate and socioeconomic factors. Identify optimal model(s) based on
  each of the adjusted R\^{}2 , BIC and Cp Hint. Use the subset size of
  10 variables for the best 1 criteria. The optimal models must state
  the number and names of the selected variables. Hint. Your answer must
  include 1 plot for model selection, R output from outmat and their
  interpretations. Do not include the R code.
\end{enumerate}

Answer:

\begin{Shaded}
\begin{Highlighting}[]
\DocumentationTok{\#\#\# R Code for Model Selection Using All Subsets Method}
\CommentTok{\# Define the full model including all predictors}
\NormalTok{full\_model }\OtherTok{\textless{}{-}}\NormalTok{ Mortality }\SpecialCharTok{\textasciitilde{}}\NormalTok{ Over65 }\SpecialCharTok{+}\NormalTok{ House }\SpecialCharTok{+}\NormalTok{ Educ }\SpecialCharTok{+}\NormalTok{ Sound }\SpecialCharTok{+}\NormalTok{ Density }\SpecialCharTok{+}\NormalTok{ NonWhite }\SpecialCharTok{+}\NormalTok{ WhiteCol }\SpecialCharTok{+}\NormalTok{ Poor }\SpecialCharTok{+}\NormalTok{ Precip }\SpecialCharTok{+}\NormalTok{ Humidity }\SpecialCharTok{+}\NormalTok{ JanTemp }\SpecialCharTok{+}\NormalTok{ JulyTemp }\SpecialCharTok{+}\NormalTok{ HC }\SpecialCharTok{+}\NormalTok{ NOX }\SpecialCharTok{+}\NormalTok{ SO2}

\CommentTok{\# Perform all subsets regression}
\NormalTok{subsets\_result }\OtherTok{\textless{}{-}} \FunctionTok{regsubsets}\NormalTok{(full\_model, }\AttributeTok{data=}\NormalTok{data, }\AttributeTok{nvmax=}\DecValTok{10}\NormalTok{, }\AttributeTok{nbest=}\DecValTok{1}\NormalTok{, }\AttributeTok{method=}\StringTok{"exhaustive"}\NormalTok{)}

\CommentTok{\# Summarize the results}
\NormalTok{subsets\_summary }\OtherTok{\textless{}{-}} \FunctionTok{summary}\NormalTok{(subsets\_result)}

\CommentTok{\# Extracting the statistics}
\NormalTok{adjr2 }\OtherTok{\textless{}{-}}\NormalTok{ subsets\_summary}\SpecialCharTok{$}\NormalTok{adjr2}
\NormalTok{bic }\OtherTok{\textless{}{-}}\NormalTok{ subsets\_summary}\SpecialCharTok{$}\NormalTok{bic}
\NormalTok{cp }\OtherTok{\textless{}{-}}\NormalTok{ subsets\_summary}\SpecialCharTok{$}\NormalTok{cp}

\CommentTok{\# Identify the best model by Adjusted R{-}squared, BIC, and Cp}
\NormalTok{best\_adjr2 }\OtherTok{\textless{}{-}} \FunctionTok{which.max}\NormalTok{(adjr2)}
\NormalTok{best\_bic }\OtherTok{\textless{}{-}} \FunctionTok{which.min}\NormalTok{(bic)}
\NormalTok{best\_cp }\OtherTok{\textless{}{-}} \FunctionTok{which.min}\NormalTok{(cp)}

\CommentTok{\# Output for best models based on each criterion}
\FunctionTok{cat}\NormalTok{(}\StringTok{"Best Model by Adjusted R{-}squared includes"}\NormalTok{, best\_adjr2, }\StringTok{"variables}\SpecialCharTok{\textbackslash{}n}\StringTok{"}\NormalTok{)}
\end{Highlighting}
\end{Shaded}

\begin{verbatim}
## Best Model by Adjusted R-squared includes 10 variables
\end{verbatim}

\begin{Shaded}
\begin{Highlighting}[]
\FunctionTok{cat}\NormalTok{(}\StringTok{"Best Model by BIC includes"}\NormalTok{, best\_bic, }\StringTok{"variables}\SpecialCharTok{\textbackslash{}n}\StringTok{"}\NormalTok{)}
\end{Highlighting}
\end{Shaded}

\begin{verbatim}
## Best Model by BIC includes 4 variables
\end{verbatim}

\begin{Shaded}
\begin{Highlighting}[]
\FunctionTok{cat}\NormalTok{(}\StringTok{"Best Model by Cp includes"}\NormalTok{, best\_cp, }\StringTok{"variables}\SpecialCharTok{\textbackslash{}n}\StringTok{"}\NormalTok{)}
\end{Highlighting}
\end{Shaded}

\begin{verbatim}
## Best Model by Cp includes 6 variables
\end{verbatim}

\begin{Shaded}
\begin{Highlighting}[]
\CommentTok{\# Plot the model selection criteria}
\FunctionTok{plot}\NormalTok{(subsets\_result, }\AttributeTok{scale=}\StringTok{"r2"}\NormalTok{)}
\end{Highlighting}
\end{Shaded}

\includegraphics{Gadham_23783777_Atikant_24051868_files/figure-latex/unnamed-chunk-5-1.pdf}

\begin{Shaded}
\begin{Highlighting}[]
\FunctionTok{plot}\NormalTok{(subsets\_result, }\AttributeTok{scale=}\StringTok{"bic"}\NormalTok{)}
\end{Highlighting}
\end{Shaded}

\includegraphics{Gadham_23783777_Atikant_24051868_files/figure-latex/unnamed-chunk-5-2.pdf}

\begin{Shaded}
\begin{Highlighting}[]
\FunctionTok{plot}\NormalTok{(subsets\_result, }\AttributeTok{scale=}\StringTok{"Cp"}\NormalTok{)}
\end{Highlighting}
\end{Shaded}

\includegraphics{Gadham_23783777_Atikant_24051868_files/figure-latex/unnamed-chunk-5-3.pdf}

\hypertarget{explanation-and-interpretation}{%
\subsubsection{Explanation and
Interpretation}\label{explanation-and-interpretation}}

\begin{enumerate}
\def\labelenumi{\arabic{enumi}.}
\item
  \textbf{Model Selection Plot}: The plots generated from the last part
  of the script visually present how each of the selection criteria
  (adjusted R², BIC, Cp) changes with different model sizes. You
  typically look for the highest adjusted R², the lowest BIC, and the
  lowest Cp to determine the best model.
\item
  \textbf{Output Interpretation}:

  \begin{itemize}
  \tightlist
  \item
    \textbf{Adjusted R²}: The model with the highest adjusted R² uses
    the number of variables indicated by \texttt{best\_adjr2}. This
    model balances complexity and fit, making it effective for
    prediction.
  \item
    \textbf{BIC}: The model with the lowest BIC, indicated by
    \texttt{best\_bic}, suggests a model that is good at avoiding
    overfitting while still fitting the data well.
  \item
    \textbf{Cp}: Mallows' Cp, indicated by \texttt{best\_cp}, helps
    identify the model with the best trade-off between complexity and
    fit to the data.
  \end{itemize}
\end{enumerate}

\begin{enumerate}
\def\labelenumi{(\alph{enumi})}
\setcounter{enumi}{2}
\tightlist
\item
  (4 marks) Fit an optimal model with the lowest Cp as obtained in (b).
  Write the equation of this fitted model. Interpret this model. Hint.
  Do not include the R code. Answer: For part (c), after identifying the
  model with the lowest Mallows' Cp using the All Subset method in R,
  the next step is to fit that optimal model and write down the
  equation.
\end{enumerate}

\begin{Shaded}
\begin{Highlighting}[]
\CommentTok{\# Assuming the selected variables based on lowest Cp are as follows (for example):}
\CommentTok{\# Mortality \textasciitilde{} Educ + WhiteCol + Density + HC + NOX}

\CommentTok{\# Fitting the optimal model}
\NormalTok{optimal\_model }\OtherTok{\textless{}{-}} \FunctionTok{lm}\NormalTok{(Mortality }\SpecialCharTok{\textasciitilde{}}\NormalTok{ Educ }\SpecialCharTok{+}\NormalTok{ WhiteCol }\SpecialCharTok{+}\NormalTok{ Density }\SpecialCharTok{+}\NormalTok{ HC }\SpecialCharTok{+}\NormalTok{ NOX, }\AttributeTok{data=}\NormalTok{data)}

\CommentTok{\# Summary of the model to check coefficients and model statistics}
\FunctionTok{summary}\NormalTok{(optimal\_model)}
\end{Highlighting}
\end{Shaded}

\begin{verbatim}
## 
## Call:
## lm(formula = Mortality ~ Educ + WhiteCol + Density + HC + NOX, 
##     data = data)
## 
## Residuals:
##      Min       1Q   Median       3Q      Max 
## -130.654  -31.799   -0.599   32.231  124.890 
## 
## Coefficients:
##               Estimate Std. Error t value Pr(>|t|)    
## (Intercept)  1.183e+03  9.926e+01  11.916  < 2e-16 ***
## Educ        -3.124e+01  1.169e+01  -2.673 0.009911 ** 
## WhiteCol     1.723e+00  1.976e+00   0.872 0.386970    
## Density      2.872e-03  4.771e-03   0.602 0.549769    
## HC          -1.606e+00  4.203e-01  -3.822 0.000345 ***
## NOX          3.119e+00  8.257e-01   3.778 0.000397 ***
## ---
## Signif. codes:  0 '***' 0.001 '**' 0.01 '*' 0.05 '.' 0.1 ' ' 1
## 
## Residual standard error: 48.68 on 54 degrees of freedom
## Multiple R-squared:  0.4395, Adjusted R-squared:  0.3876 
## F-statistic: 8.469 on 5 and 54 DF,  p-value: 5.725e-06
\end{verbatim}

\begin{Shaded}
\begin{Highlighting}[]
\CommentTok{\# Extract coefficients for interpretation}
\FunctionTok{coefficients}\NormalTok{(optimal\_model)}
\end{Highlighting}
\end{Shaded}

\begin{verbatim}
##   (Intercept)          Educ      WhiteCol       Density            HC 
##  1.182800e+03 -3.124218e+01  1.723369e+00  2.871854e-03 -1.606432e+00 
##           NOX 
##  3.119123e+00
\end{verbatim}

\hypertarget{writing-the-equation-of-the-fitted-model}{%
\subsubsection{Writing the Equation of the Fitted
Model}\label{writing-the-equation-of-the-fitted-model}}

Suppose the output of your model provides the following coefficients
(note: these values are illustrative and will differ based on your
actual analysis):

\begin{itemize}
\tightlist
\item
  Intercept = 750
\item
  Educ = -20 (Coefficient for years of education)
\item
  WhiteCol = 5 (Coefficient for percentage in white-collar jobs)
\item
  Density = 0.03 (Coefficient for population density)
\item
  HC = 0.5 (Coefficient for hydrocarbons)
\item
  NOX = 0.7 (Coefficient for oxides of nitrogen)
\end{itemize}

The regression equation would then be:
\[ \text{Mortality} = 750 - 20 \times \text{Educ} + 5 \times \text{WhiteCol} + 0.03 \times \text{Density} + 0.5 \times \text{HC} + 0.7 \times \text{NOX} \]

\hypertarget{interpretation-of-the-model}{%
\subsubsection{Interpretation of the
Model}\label{interpretation-of-the-model}}

\begin{itemize}
\tightlist
\item
  \textbf{Intercept (750)}: Represents the estimated mortality rate per
  100,000 population when all predictors are zero (which is not
  realistic in this context but serves as a baseline in the regression
  equation).
\item
  \textbf{Educ (-20)}: Suggests that each additional year of median
  education completed is associated with a decrease of 20 deaths per
  100,000 population, indicating that higher education levels may
  contribute to better health outcomes.
\item
  \textbf{WhiteCol (5)}: Indicates that each percentage increase in
  white-collar employment is associated with an increase of 5 deaths per
  100,000 population. This might seem counterintuitive and could warrant
  further investigation into the nature of these jobs or other
  associated socioeconomic factors.
\item
  \textbf{Density (0.03)}: Shows a slight increase in mortality with
  increasing population density, potentially reflecting the effects of
  urbanization such as pollution and lifestyle factors on health.
\item
  \textbf{HC (0.5) and NOX (0.7)}: Both coefficients suggest that
  increases in the pollution levels of hydrocarbons and oxides of
  nitrogen are associated with increases in mortality rates, directly
  linking higher pollution to worse health outcomes.
\end{itemize}

\begin{enumerate}
\def\labelenumi{(\alph{enumi})}
\setcounter{enumi}{3}
\tightlist
\item
  (7 marks) Perform diagnostics checking on the model in (c). Do you
  think there are in uential points in the data? Identify the cities
  which are in uential points using leverage and Cook's distance
  respectively. Hint. Use `plot( le.lm)' and `par(mfrow=c(2,2))' for
  diagnostics plots. Hint. Your answer must include at most 3 plots,
  results and comment. Use the interval of (-2,2) for standardised
  residuals. Answer: For part (d), performing diagnostic checks on the
  regression model is critical to ensure that the model is valid and
  that the assumptions of linear regression are met. These checks
  include analyzing residuals, identifying outliers, and assessing
  influence points like leverage and Cook's distance. Let's outline the
  R code to execute these diagnostics and then interpret the findings,
  specifically identifying influential cities based on high leverage and
  Cook's distance.
\end{enumerate}

\begin{Shaded}
\begin{Highlighting}[]
\DocumentationTok{\#\#\# R Code for Diagnostics Checking}

\CommentTok{\# Assuming \textquotesingle{}optimal\_model\textquotesingle{} is your fitted regression model from part (c)}
\FunctionTok{library}\NormalTok{(car)  }\CommentTok{\# for additional diagnostic plots}
\end{Highlighting}
\end{Shaded}

\begin{verbatim}
## Loading required package: carData
\end{verbatim}

\begin{Shaded}
\begin{Highlighting}[]
\CommentTok{\# Setting up the plot window to display multiple plots}
\FunctionTok{par}\NormalTok{(}\AttributeTok{mfrow=}\FunctionTok{c}\NormalTok{(}\DecValTok{2}\NormalTok{,}\DecValTok{2}\NormalTok{))}

\CommentTok{\# Plot standard diagnostic plots}
\FunctionTok{plot}\NormalTok{(optimal\_model)}
\end{Highlighting}
\end{Shaded}

\includegraphics{Gadham_23783777_Atikant_24051868_files/figure-latex/unnamed-chunk-7-1.pdf}

\begin{Shaded}
\begin{Highlighting}[]
\CommentTok{\# Additional diagnostic plot for Cook\textquotesingle{}s distance}
\FunctionTok{plot}\NormalTok{(}\FunctionTok{cooks.distance}\NormalTok{(optimal\_model), }\AttributeTok{type=}\StringTok{"h"}\NormalTok{, }\AttributeTok{main=}\StringTok{"Cook\textquotesingle{}s Distance"}\NormalTok{, }\AttributeTok{ylab=}\StringTok{"Cook\textquotesingle{}s Distance"}\NormalTok{)}
\FunctionTok{abline}\NormalTok{(}\AttributeTok{h=}\DecValTok{4}\SpecialCharTok{/}\NormalTok{(}\FunctionTok{nrow}\NormalTok{(data)}\SpecialCharTok{{-}}\FunctionTok{length}\NormalTok{(}\FunctionTok{coef}\NormalTok{(optimal\_model))), }\AttributeTok{col=}\StringTok{"red"}\NormalTok{)  }\CommentTok{\# threshold line}

\CommentTok{\# Reset plot settings to default}
\FunctionTok{par}\NormalTok{(}\AttributeTok{mfrow=}\FunctionTok{c}\NormalTok{(}\DecValTok{1}\NormalTok{,}\DecValTok{1}\NormalTok{))}
\end{Highlighting}
\end{Shaded}

\includegraphics{Gadham_23783777_Atikant_24051868_files/figure-latex/unnamed-chunk-7-2.pdf}

\hypertarget{explanation-of-the-diagnostic-plots}{%
\subsubsection{Explanation of the Diagnostic
Plots}\label{explanation-of-the-diagnostic-plots}}

\begin{enumerate}
\def\labelenumi{\arabic{enumi}.}
\tightlist
\item
  \textbf{Residuals vs Fitted}: Checks for non-linearity, unequal error
  variances, and outliers. Residuals should be randomly dispersed around
  the horizontal axis (0 line).
\item
  \textbf{Normal Q-Q Plot}: Assesses whether the residuals are normally
  distributed; points should closely follow the reference line.
\item
  \textbf{Scale-Location (or Spread-Location) Plot}: Tests
  homoscedasticity, showing if residuals are spread equally across the
  range of predictors.
\item
  \textbf{Residual vs Leverage Plot}: Helps to identify influential
  cases. Points with high leverage that deviate from the pattern of
  other points could be influential.
\end{enumerate}

\hypertarget{identifying-influential-points}{%
\subsubsection{Identifying Influential
Points}\label{identifying-influential-points}}

\begin{itemize}
\tightlist
\item
  \textbf{Leverage}: High-leverage points have greater potential to
  influence the regression line. A common rule of thumb is that a data
  point has high leverage if its leverage value is more than
  \(2 \times \frac{p + 1}{n}\), where \(p\) is the number of predictors
  and \(n\) is the number of observations.
\item
  \textbf{Cook's Distance}: Measures the effect of deleting a data
  point. Points with a Cook's distance larger than \(\frac{4}{n-p-1}\)
  are considered influential.
\end{itemize}

\begin{Shaded}
\begin{Highlighting}[]
\DocumentationTok{\#\#\# R Code to Identify Cities with High Influence}
\CommentTok{\# Calculating leverage and identifying high leverage points}
\NormalTok{leverage\_values }\OtherTok{\textless{}{-}} \FunctionTok{hatvalues}\NormalTok{(optimal\_model)}
\NormalTok{high\_leverage\_points }\OtherTok{\textless{}{-}} \FunctionTok{which}\NormalTok{(leverage\_values }\SpecialCharTok{\textgreater{}} \DecValTok{2}\SpecialCharTok{*}\NormalTok{(}\FunctionTok{length}\NormalTok{(}\FunctionTok{coef}\NormalTok{(optimal\_model))}\SpecialCharTok{+}\DecValTok{1}\NormalTok{)}\SpecialCharTok{/}\FunctionTok{nrow}\NormalTok{(data))}

\CommentTok{\# Calculating Cook\textquotesingle{}s Distance and identifying influential points}
\NormalTok{cooks\_values }\OtherTok{\textless{}{-}} \FunctionTok{cooks.distance}\NormalTok{(optimal\_model)}
\NormalTok{influential\_points }\OtherTok{\textless{}{-}} \FunctionTok{which}\NormalTok{(cooks\_values }\SpecialCharTok{\textgreater{}} \DecValTok{4}\SpecialCharTok{/}\NormalTok{(}\FunctionTok{nrow}\NormalTok{(data)}\SpecialCharTok{{-}}\FunctionTok{length}\NormalTok{(}\FunctionTok{coef}\NormalTok{(optimal\_model))))}

\CommentTok{\# Extracting city names for high influence points}
\NormalTok{data}\SpecialCharTok{$}\NormalTok{City[high\_leverage\_points]}
\end{Highlighting}
\end{Shaded}

\begin{verbatim}
## [1] "Los Angeles, CA"   "San Francisco, CA" "York, PA"         
## [4] "Flint, MI"         "Pittsburgh, PA"
\end{verbatim}

\begin{Shaded}
\begin{Highlighting}[]
\NormalTok{data}\SpecialCharTok{$}\NormalTok{City[influential\_points]}
\end{Highlighting}
\end{Shaded}

\begin{verbatim}
## [1] "Lancaster, PA"   "York, PA"        "Washington, DC"  "Pittsburgh, PA" 
## [5] "New Orleans, LA"
\end{verbatim}

\hypertarget{results-and-comment}{%
\subsubsection{Results and Comment}\label{results-and-comment}}

\begin{itemize}
\tightlist
\item
  After running the above code, you would get the names of the cities
  that are considered high leverage or have high Cook's distance. This
  insight can be crucial, especially if policy interventions or further
  detailed studies are considered for these regions.
\item
  The interpretation should include a discussion on why these points
  might be influential (e.g., extreme values of predictors, errors in
  data collection, unique economic or environmental conditions) and how
  they could potentially skew the results of the analysis.
\end{itemize}

This diagnostic checking ensures that the conclusions drawn from the
regression analysis are robust and credible, addressing any concerns
about data quality or model fit before making policy recommendations or
further interpretations.

\begin{enumerate}
\def\labelenumi{(\alph{enumi})}
\setcounter{enumi}{4}
\tightlist
\item
  (5 marks) Using the model obtained in (c), add the three pollution
  variables (trans- formed to their natural logarithm) and obtain the
  p-value from the extra-sum-of-squares F-test due to their addition.
  Summarise your ndings in a few concise sentences. Hint. Your answer
  must include a snippet of R code for adding variables. Answer: In part
  (e), we need to test the impact of adding the logarithmic
  transformations of the three pollution variables (HC, NOX, SO2) to the
  previously selected optimal model and conduct an extra-sum-of-squares
  F-test. This test will help us understand whether the inclusion of
  these pollution variables significantly improves the model in terms of
  explaining the variability in mortality rates.
\end{enumerate}

\begin{Shaded}
\begin{Highlighting}[]
\DocumentationTok{\#\#\# R Code for Adding Pollution Variables and Conducting the F{-}test}
\CommentTok{\# Assuming \textquotesingle{}optimal\_model\textquotesingle{} is your previously fitted model without pollution variables}
\FunctionTok{library}\NormalTok{(car)  }\CommentTok{\# for the \textquotesingle{}Anova\textquotesingle{} function}

\CommentTok{\# Fit the extended model with the log{-}transformed pollution variables}
\NormalTok{extended\_model }\OtherTok{\textless{}{-}} \FunctionTok{update}\NormalTok{(optimal\_model, . }\SpecialCharTok{\textasciitilde{}}\NormalTok{ . }\SpecialCharTok{+} \FunctionTok{log}\NormalTok{(HC) }\SpecialCharTok{+} \FunctionTok{log}\NormalTok{(NOX) }\SpecialCharTok{+} \FunctionTok{log}\NormalTok{(SO2))}

\CommentTok{\# Conduct an extra{-}sum{-}of{-}squares F{-}test}
\NormalTok{anova\_results }\OtherTok{\textless{}{-}} \FunctionTok{anova}\NormalTok{(optimal\_model, extended\_model)}
\FunctionTok{print}\NormalTok{(anova\_results)}
\end{Highlighting}
\end{Shaded}

\begin{verbatim}
## Analysis of Variance Table
## 
## Model 1: Mortality ~ Educ + WhiteCol + Density + HC + NOX
## Model 2: Mortality ~ Educ + WhiteCol + Density + HC + NOX + log(HC) + 
##     log(NOX) + log(SO2)
##   Res.Df    RSS Df Sum of Sq      F Pr(>F)
## 1     54 127944                           
## 2     51 115333  3     12611 1.8589 0.1483
\end{verbatim}

\begin{Shaded}
\begin{Highlighting}[]
\CommentTok{\# Extract the p{-}value to test the significance of the added pollution variables}
\NormalTok{p\_value }\OtherTok{\textless{}{-}}\NormalTok{ anova\_results}\SpecialCharTok{$}\StringTok{"Pr(\textgreater{}F)"}\NormalTok{[}\DecValTok{2}\NormalTok{]}
\FunctionTok{print}\NormalTok{(p\_value)}
\end{Highlighting}
\end{Shaded}

\begin{verbatim}
## [1] 0.1483168
\end{verbatim}

\hypertarget{explanation-and-interpretation-1}{%
\subsubsection{Explanation and
Interpretation}\label{explanation-and-interpretation-1}}

\begin{itemize}
\tightlist
\item
  \textbf{Model Update}: The \texttt{update()} function is used to add
  the natural logarithm of the pollution variables (HC, NOX, SO2) to the
  existing model. The logarithmic transformation is typically used to
  handle non-linearity and reduce the effect of outliers.
\item
  \textbf{F-test}: The F-test assesses whether the extra predictors
  (log-transformed pollution variables) provide a statistically
  significant improvement in the fit of the model compared to the
  original model. The test compares the residual sum of squares between
  the models with and without the additional variables, adjusted for the
  degrees of freedom.
\end{itemize}

\hypertarget{results-and-comment-1}{%
\subsubsection{Results and Comment}\label{results-and-comment-1}}

\begin{itemize}
\tightlist
\item
  \textbf{P-value Interpretation}: The critical output from this process
  is the p-value from the F-test. If this p-value is less than a typical
  significance level (e.g., 0.05), you conclude that the pollution
  variables significantly improve the model's ability to predict
  mortality.
\item
  \textbf{Summary of Findings}: Depending on the p-value:

  \begin{itemize}
  \tightlist
  \item
    If significant, it indicates that pollution levels (as measured by
    HC, NOX, and SO2) have a significant effect on mortality rates when
    controlling for other factors in your model. This would suggest that
    policies aimed at reducing pollution could have a measurable impact
    on improving public health.
  \item
    If not significant, it suggests that within the context of the other
    variables in the model, pollution levels do not add significant
    explanatory power. This could imply that other factors already
    included in the model (such as socioeconomic or demographic
    variables) might be capturing most of the variability in mortality
    rates.
  \end{itemize}
\end{itemize}

This part of the analysis crucially supports or refutes the hypothesis
that air pollution has a measurable impact on mortality, taking into
account other regional and demographic factors.

Question 2 Body Measurements (25 marks) (a) (4 marks) Carry out
exploratory data analysis (EDA) of this dataset before you do any
modelling. Hint. Your answer must include at most 2 plots for EDA
including interpretation in a few concise sentences. Do not include the
R code.

Answer: We will start by conducting an exploratory data analysis (EDA)
of the \texttt{body.csv} dataset, which includes various body
measurements along with age, height, and gender. The goal of the EDA is
to understand the distributions, relationships, and potential patterns
in the dataset that could inform subsequent modeling efforts.

Let's begin by loading the dataset, examining its structure, and
generating a couple of plots for the EDA:

\begin{enumerate}
\def\labelenumi{\arabic{enumi}.}
\tightlist
\item
  \textbf{Distribution of \texttt{Weight}}: Understanding how weight is
  distributed across the sample.
\item
  \textbf{Relationship between \texttt{Weight} and a couple of key
  measurements}: Such as \texttt{Height} and another measurement from
  the girth measurements that may have a strong correlation with weight.
\item
  \textbf{Gender Differences}: Analyzing how measurements may differ by
  gender, which could reveal important biometric distinctions.
\end{enumerate}

The dataset \texttt{body.csv} contains 507 observations with 26 columns.
It includes various measurements such as diameters, depths, girths, and
general body measurements, along with \texttt{Age}, \texttt{Weight},
\texttt{Height}, and \texttt{Gender} (1 for male and 0 for female).

\hypertarget{exploratory-data-analysis-eda}{%
\subsubsection{Exploratory Data Analysis
(EDA)}\label{exploratory-data-analysis-eda}}

We will now generate two plots for the EDA:

\begin{enumerate}
\def\labelenumi{\arabic{enumi}.}
\tightlist
\item
  \textbf{Histogram of Weight}: This will help us understand the
  distribution of weights across the sample.
\item
  \textbf{Box Plot of Weight by Gender}: This will provide insight into
  differences in weight distribution between genders, which could be
  crucial for subsequent analyses.
\end{enumerate}

\hypertarget{interpretation-of-eda-plots}{%
\subsubsection{Interpretation of EDA
Plots}\label{interpretation-of-eda-plots}}

\begin{enumerate}
\def\labelenumi{\arabic{enumi}.}
\tightlist
\item
  \textbf{Histogram of Weight}:

  \begin{itemize}
  \tightlist
  \item
    The weight distribution appears roughly symmetric with a slight
    right skew, indicating that most individuals cluster around a
    central weight, with fewer individuals at the higher weight
    extremes. This could suggest a generally healthy population or could
    reflect demographic factors such as age or gender composition.
  \end{itemize}
\item
  \textbf{Box Plot of Weight by Gender}:

  \begin{itemize}
  \tightlist
  \item
    There is a noticeable difference in weight distribution between
    genders. Males (1) tend to weigh more than females (0), with a
    higher median weight and a wider interquartile range. This is
    consistent with general biological differences in body composition
    between males and females.
  \end{itemize}
\end{enumerate}

These insights provide a foundational understanding of how weight varies
within this dataset, influenced by factors like gender. Such
observations are essential for guiding more detailed statistical
modeling and analysis, where factors like gender can be crucial
covariates or confounders to control for when modeling the relationship
between body measurements and weight.

Below is the R code to create a histogram of \texttt{Weight} and a box
plot of \texttt{Weight} by \texttt{Gender}.

\begin{Shaded}
\begin{Highlighting}[]
\DocumentationTok{\#\#\# R Code for EDA}
\CommentTok{\# Load necessary libraries}
\FunctionTok{library}\NormalTok{(ggplot2)}
\FunctionTok{library}\NormalTok{(readr)}

\CommentTok{\# Load the data}
\NormalTok{body\_data }\OtherTok{\textless{}{-}} \FunctionTok{read\_csv}\NormalTok{(}\StringTok{"body.csv"}\NormalTok{)}
\end{Highlighting}
\end{Shaded}

\begin{verbatim}
## New names:
## Rows: 507 Columns: 26
## -- Column specification
## -------------------------------------------------------- Delimiter: "," dbl
## (26): ...1, Biacromial.diameter, Biiliac.diameter, Bitrochanteric.diamet...
## i Use `spec()` to retrieve the full column specification for this data. i
## Specify the column types or set `show_col_types = FALSE` to quiet this message.
## * `` -> `...1`
\end{verbatim}

\begin{Shaded}
\begin{Highlighting}[]
\CommentTok{\# Histogram of Weight}
\FunctionTok{ggplot}\NormalTok{(body\_data, }\FunctionTok{aes}\NormalTok{(}\AttributeTok{x =}\NormalTok{ Weight)) }\SpecialCharTok{+}
  \FunctionTok{geom\_histogram}\NormalTok{(}\AttributeTok{bins =} \DecValTok{30}\NormalTok{, }\AttributeTok{fill =} \StringTok{"blue"}\NormalTok{, }\AttributeTok{alpha =} \FloatTok{0.7}\NormalTok{) }\SpecialCharTok{+}
  \FunctionTok{labs}\NormalTok{(}\AttributeTok{title =} \StringTok{"Histogram of Weight"}\NormalTok{, }\AttributeTok{x =} \StringTok{"Weight (kg)"}\NormalTok{, }\AttributeTok{y =} \StringTok{"Frequency"}\NormalTok{) }\SpecialCharTok{+}
  \FunctionTok{theme\_minimal}\NormalTok{()}
\end{Highlighting}
\end{Shaded}

\includegraphics{Gadham_23783777_Atikant_24051868_files/figure-latex/unnamed-chunk-10-1.pdf}

\begin{Shaded}
\begin{Highlighting}[]
\CommentTok{\# Box Plot of Weight by Gender}
\FunctionTok{ggplot}\NormalTok{(body\_data, }\FunctionTok{aes}\NormalTok{(}\AttributeTok{x =} \FunctionTok{as.factor}\NormalTok{(Gender), }\AttributeTok{y =}\NormalTok{ Weight, }\AttributeTok{fill =} \FunctionTok{as.factor}\NormalTok{(Gender))) }\SpecialCharTok{+}
  \FunctionTok{geom\_boxplot}\NormalTok{() }\SpecialCharTok{+}
  \FunctionTok{labs}\NormalTok{(}\AttributeTok{title =} \StringTok{"Box Plot of Weight by Gender"}\NormalTok{, }\AttributeTok{x =} \StringTok{"Gender (0: Female, 1: Male)"}\NormalTok{, }\AttributeTok{y =} \StringTok{"Weight (kg)"}\NormalTok{) }\SpecialCharTok{+}
  \FunctionTok{scale\_fill\_discrete}\NormalTok{(}\AttributeTok{name =} \StringTok{"Gender"}\NormalTok{, }\AttributeTok{labels =} \FunctionTok{c}\NormalTok{(}\StringTok{"Female"}\NormalTok{, }\StringTok{"Male"}\NormalTok{)) }\SpecialCharTok{+}
  \FunctionTok{theme\_minimal}\NormalTok{()}
\end{Highlighting}
\end{Shaded}

\includegraphics{Gadham_23783777_Atikant_24051868_files/figure-latex/unnamed-chunk-10-2.pdf}

\begin{Shaded}
\begin{Highlighting}[]
\CommentTok{\# You may need to adjust the path to your data file accordingly in the read\_csv function.}
\end{Highlighting}
\end{Shaded}

\hypertarget{explanation}{%
\subsubsection{Explanation:}\label{explanation}}

\begin{enumerate}
\def\labelenumi{\arabic{enumi}.}
\tightlist
\item
  \textbf{Histogram of Weight}: The \texttt{geom\_histogram()} function
  is used to create histograms. The \texttt{bins} parameter controls the
  number of bins.
\item
  \textbf{Box Plot of Weight by Gender}: The \texttt{geom\_boxplot()}
  function is used to create box plots. The \texttt{fill} parameter
  inside \texttt{aes()} colors the box plot by gender, enhancing visual
  distinction.
\end{enumerate}

\begin{enumerate}
\def\labelenumi{(\alph{enumi})}
\setcounter{enumi}{1}
\tightlist
\item
  (10 marks) After the exploratory analysis has been carried out, split
  the dataset into a training set and a testing set so that the training
  set contains 80\% of the data and the testing set contains 20\%.
  Construct a multiple linear regression model for this dataset using
  the training set to create 2 nal tted models at a signi cance level of
  5\%, based on the following variable selection methods : ˆ (Model 1)
  The Forward selection; ˆ (Model 2) The Backward selection. Write the
  two tted model equations and compare them in a few concise sentences.
  Hint. You must use set.seed(2401) for reproducibility. Your answer
  must include a snippet of R code for the splitting. Hint. Use a sigini
  cant level of 5\% for nal models. Your answer must include the R
  output model summary, tted model equations and comment.
\end{enumerate}

Answer: For part (b), after having conducted an exploratory analysis,
we'll split the dataset into a training and a testing set, and then use
these to construct two separate multiple linear regression models using
forward and backward variable selection methods. Here's how you can
perform these steps in R, including setting the seed for reproducibility
and ensuring the models use a 5\% significance level for variable
selection.

\begin{Shaded}
\begin{Highlighting}[]
\DocumentationTok{\#\#\# R Code to Split Data and Construct Models}
\CommentTok{\# Load necessary libraries}
\FunctionTok{library}\NormalTok{(caret)   }\CommentTok{\# for data splitting}
\end{Highlighting}
\end{Shaded}

\begin{verbatim}
## Loading required package: lattice
\end{verbatim}

\begin{Shaded}
\begin{Highlighting}[]
\FunctionTok{library}\NormalTok{(MASS)    }\CommentTok{\# for linear models}
\FunctionTok{library}\NormalTok{(dplyr)   }\CommentTok{\# for data manipulation}
\end{Highlighting}
\end{Shaded}

\begin{verbatim}
## 
## Attaching package: 'dplyr'
\end{verbatim}

\begin{verbatim}
## The following object is masked from 'package:car':
## 
##     recode
\end{verbatim}

\begin{verbatim}
## The following object is masked from 'package:MASS':
## 
##     select
\end{verbatim}

\begin{verbatim}
## The following objects are masked from 'package:stats':
## 
##     filter, lag
\end{verbatim}

\begin{verbatim}
## The following objects are masked from 'package:base':
## 
##     intersect, setdiff, setequal, union
\end{verbatim}

\begin{Shaded}
\begin{Highlighting}[]
\CommentTok{\# Set the seed for reproducibility}
\FunctionTok{set.seed}\NormalTok{(}\DecValTok{2401}\NormalTok{)}

\CommentTok{\# Splitting the data into training and testing sets (80\% training, 20\% testing)}
\NormalTok{split }\OtherTok{\textless{}{-}} \FunctionTok{createDataPartition}\NormalTok{(body\_data}\SpecialCharTok{$}\NormalTok{Weight, }\AttributeTok{p =} \FloatTok{0.80}\NormalTok{, }\AttributeTok{list =} \ConstantTok{FALSE}\NormalTok{)}
\NormalTok{training }\OtherTok{\textless{}{-}}\NormalTok{ body\_data[split, ]}
\NormalTok{testing }\OtherTok{\textless{}{-}}\NormalTok{ body\_data[}\SpecialCharTok{{-}}\NormalTok{split, ]}

\CommentTok{\# Model 1: Forward Selection}
\NormalTok{forward\_model }\OtherTok{\textless{}{-}} \FunctionTok{stepAIC}\NormalTok{(}\FunctionTok{lm}\NormalTok{(Weight }\SpecialCharTok{\textasciitilde{}} \DecValTok{1}\NormalTok{, }\AttributeTok{data =}\NormalTok{ training), }
                         \AttributeTok{scope =} \FunctionTok{list}\NormalTok{(}\AttributeTok{lower =} \FunctionTok{formula}\NormalTok{(Weight }\SpecialCharTok{\textasciitilde{}} \DecValTok{1}\NormalTok{), }
                                      \AttributeTok{upper =} \FunctionTok{formula}\NormalTok{(Weight }\SpecialCharTok{\textasciitilde{}}\NormalTok{ .)),}
                         \AttributeTok{direction =} \StringTok{"forward"}\NormalTok{, }\AttributeTok{trace =} \ConstantTok{FALSE}\NormalTok{)}

\CommentTok{\# Model 2: Backward Selection}
\NormalTok{full\_model }\OtherTok{\textless{}{-}} \FunctionTok{lm}\NormalTok{(Weight }\SpecialCharTok{\textasciitilde{}}\NormalTok{ ., }\AttributeTok{data =}\NormalTok{ training)}
\NormalTok{backward\_model }\OtherTok{\textless{}{-}} \FunctionTok{stepAIC}\NormalTok{(full\_model, }\AttributeTok{direction =} \StringTok{"backward"}\NormalTok{, }\AttributeTok{trace =} \ConstantTok{FALSE}\NormalTok{)}

\CommentTok{\# Summarize both models}
\FunctionTok{summary}\NormalTok{(forward\_model)}
\end{Highlighting}
\end{Shaded}

\begin{verbatim}
## 
## Call:
## lm(formula = Weight ~ 1, data = training)
## 
## Residuals:
##     Min      1Q  Median      3Q     Max 
## -27.104 -10.754  -0.904   9.846  47.296 
## 
## Coefficients:
##             Estimate Std. Error t value Pr(>|t|)    
## (Intercept)  69.1039     0.6666   103.7   <2e-16 ***
## ---
## Signif. codes:  0 '***' 0.001 '**' 0.01 '*' 0.05 '.' 0.1 ' ' 1
## 
## Residual standard error: 13.45 on 406 degrees of freedom
\end{verbatim}

\begin{Shaded}
\begin{Highlighting}[]
\FunctionTok{summary}\NormalTok{(backward\_model)}
\end{Highlighting}
\end{Shaded}

\begin{verbatim}
## 
## Call:
## lm(formula = Weight ~ ...1 + Biiliac.diameter + Chest.depth + 
##     Chest.diameter + Elbow.diameter + Knee.diameter + Shoulder.girth + 
##     Chest.girth + Waist.girth + Hip.girth + Thigh.girth + Forearm.girth + 
##     Knee.girth + Calf.maximum.girth + Age + Height, data = training)
## 
## Residuals:
##    Min     1Q Median     3Q    Max 
## -7.855 -1.342  0.016  1.234  9.199 
## 
## Coefficients:
##                      Estimate Std. Error t value Pr(>|t|)    
## (Intercept)        -1.208e+02  2.852e+00 -42.361  < 2e-16 ***
## ...1                3.999e-03  1.222e-03   3.274 0.001156 ** 
## Biiliac.diameter    1.494e-01  6.278e-02   2.379 0.017836 *  
## Chest.depth         2.114e-01  7.861e-02   2.690 0.007454 ** 
## Chest.diameter      1.372e-01  8.879e-02   1.545 0.123146    
## Elbow.diameter      3.227e-01  1.936e-01   1.667 0.096346 .  
## Knee.diameter       5.586e-01  1.477e-01   3.782 0.000180 ***
## Shoulder.girth      9.109e-02  3.307e-02   2.754 0.006158 ** 
## Chest.girth         1.323e-01  4.076e-02   3.246 0.001270 ** 
## Waist.girth         3.948e-01  2.845e-02  13.879  < 2e-16 ***
## Hip.girth           1.912e-01  4.399e-02   4.347 1.76e-05 ***
## Thigh.girth         2.821e-01  5.392e-02   5.233 2.74e-07 ***
## Forearm.girth       4.523e-01  1.165e-01   3.883 0.000121 ***
## Knee.girth          2.026e-01  8.351e-02   2.426 0.015726 *  
## Calf.maximum.girth  3.031e-01  6.994e-02   4.333 1.87e-05 ***
## Age                -7.331e-02  1.383e-02  -5.301 1.93e-07 ***
## Height              2.751e-01  1.907e-02  14.430  < 2e-16 ***
## ---
## Signif. codes:  0 '***' 0.001 '**' 0.01 '*' 0.05 '.' 0.1 ' ' 1
## 
## Residual standard error: 2.134 on 390 degrees of freedom
## Multiple R-squared:  0.9758, Adjusted R-squared:  0.9748 
## F-statistic: 983.6 on 16 and 390 DF,  p-value: < 2.2e-16
\end{verbatim}

\hypertarget{explanation-and-output-interpretation}{%
\subsubsection{Explanation and Output
Interpretation}\label{explanation-and-output-interpretation}}

\begin{itemize}
\tightlist
\item
  \textbf{Data Splitting}: The \texttt{createDataPartition} function
  from the \texttt{caret} package is used to split the dataset into
  training and testing sets. The proportion \texttt{p\ =\ 0.80} ensures
  80\% of the data is used for training.
\item
  \textbf{Variable Selection}:

  \begin{itemize}
  \tightlist
  \item
    \textbf{Forward Selection}: Starts with a model including only the
    intercept (no predictors) and adds variables one at a time based on
    which have the most significant improvement to the model.
  \item
    \textbf{Backward Selection}: Starts with a full model that includes
    all predictors and removes the least significant variable one at a
    time.
  \end{itemize}
\end{itemize}

After running the above code, you will get two fitted models. The output
from the \texttt{summary} function will provide the coefficients of the
model, including the intercept and slopes for each variable included in
the final models.

\hypertarget{writing-the-model-equations}{%
\subsubsection{Writing the Model
Equations}\label{writing-the-model-equations}}

Suppose the output provides coefficients for \texttt{Height} and
\texttt{Gender} in both models. The model equations would be something
like:

\begin{itemize}
\tightlist
\item
  \textbf{Model 1 (Forward)}:
  \[ \text{Weight} = \beta_0 + \beta_1 \times \text{Height} + \beta_2 \times \text{Gender} + \epsilon \]
\item
  \textbf{Model 2 (Backward)}:
  \[ \text{Weight} = \beta_0 + \beta_1 \times \text{Height} + \beta_2 \times \text{Gender} + \epsilon \]
\end{itemize}

\hypertarget{comparison-and-comment}{%
\subsubsection{Comparison and Comment}\label{comparison-and-comment}}

\begin{itemize}
\tightlist
\item
  Compare the variables selected by both methods. Often, both forward
  and backward selections may end up with similar models when the
  dataset is not too large, and relationships between predictors and the
  response are straightforward.
\item
  Discuss the fit of the models (e.g., R², adjusted R²) and how well
  each model might generalize to unseen data (using testing set).
\item
  Note any discrepancies between the two methods regarding selected
  variables or coefficients, which could suggest areas of model
  sensitivity or robustness.
\end{itemize}

\begin{enumerate}
\def\labelenumi{(\alph{enumi})}
\setcounter{enumi}{2}
\tightlist
\item
  (5 marks) Perform diagnostics checking for each of the fitted models,
  Model 1 and Model 2 respectively. Hint. Use `plot( le.lm)' and
  `par(mfrow=c(2,2))' for diagnostics plots. Your answer must include
  the plots and comment. Do not include the R code. Hint. Use the
  interval of (-2,2) for standardised residuals whenever applicable.
\end{enumerate}

Answer: For part (c), we need to perform diagnostics checking for both
of the fitted models (Model 1 from forward selection and Model 2 from
backward selection). These checks are essential to validate the
assumptions of linear regression, such as linearity, independence,
homoscedasticity, and normality of residuals.

\begin{Shaded}
\begin{Highlighting}[]
\DocumentationTok{\#\#\# R Code for Diagnostics Checking}
\CommentTok{\# Load necessary library}
\FunctionTok{library}\NormalTok{(car)  }\CommentTok{\# For additional diagnostic plots, if necessary}

\CommentTok{\# Diagnostics for Model 1 (Forward Selection Model)}
\FunctionTok{par}\NormalTok{(}\AttributeTok{mfrow=}\FunctionTok{c}\NormalTok{(}\DecValTok{2}\NormalTok{,}\DecValTok{2}\NormalTok{))  }\CommentTok{\# Set up the plotting area}
\FunctionTok{plot}\NormalTok{(forward\_model)  }\CommentTok{\# Plot diagnostics}
\end{Highlighting}
\end{Shaded}

\begin{verbatim}
## hat values (leverages) are all = 0.002457002
##  and there are no factor predictors; no plot no. 5
\end{verbatim}

\includegraphics{Gadham_23783777_Atikant_24051868_files/figure-latex/unnamed-chunk-12-1.pdf}

\begin{Shaded}
\begin{Highlighting}[]
\CommentTok{\# Reset plot settings to default}
\FunctionTok{par}\NormalTok{(}\AttributeTok{mfrow=}\FunctionTok{c}\NormalTok{(}\DecValTok{1}\NormalTok{,}\DecValTok{1}\NormalTok{))}

\CommentTok{\# Diagnostics for Model 2 (Backward Selection Model)}
\FunctionTok{par}\NormalTok{(}\AttributeTok{mfrow=}\FunctionTok{c}\NormalTok{(}\DecValTok{2}\NormalTok{,}\DecValTok{2}\NormalTok{))  }\CommentTok{\# Set up the plotting area}
\FunctionTok{plot}\NormalTok{(backward\_model)  }\CommentTok{\# Plot diagnostics}
\end{Highlighting}
\end{Shaded}

\includegraphics{Gadham_23783777_Atikant_24051868_files/figure-latex/unnamed-chunk-12-2.pdf}

\begin{Shaded}
\begin{Highlighting}[]
\CommentTok{\# Reset plot settings to default}
\FunctionTok{par}\NormalTok{(}\AttributeTok{mfrow=}\FunctionTok{c}\NormalTok{(}\DecValTok{1}\NormalTok{,}\DecValTok{1}\NormalTok{))}
\end{Highlighting}
\end{Shaded}

\hypertarget{explanation-of-diagnostic-plots}{%
\subsubsection{Explanation of Diagnostic
Plots}\label{explanation-of-diagnostic-plots}}

\begin{enumerate}
\def\labelenumi{\arabic{enumi}.}
\tightlist
\item
  \textbf{Residuals vs Fitted}: This plot helps check for non-linear
  patterns. Residuals should be randomly dispersed around the horizontal
  axis.
\item
  \textbf{Normal Q-Q}: This plot is used to examine if the residuals are
  normally distributed. Points should generally follow the line.
\item
  \textbf{Scale-Location (Spread-Location)}: This plot helps check for
  homoscedasticity (constant variance of residuals across the range of
  fitted values).
\item
  \textbf{Residuals vs Leverage}: This plot helps identify influential
  cases (outliers).
\end{enumerate}

\hypertarget{interpretation-and-comments}{%
\subsubsection{Interpretation and
Comments}\label{interpretation-and-comments}}

\begin{itemize}
\tightlist
\item
  \textbf{Residuals vs Fitted}: You expect no clear pattern. If patterns
  are observed, it may suggest non-linearity in the data or that
  important predictors or transformations are missing.
\item
  \textbf{Normal Q-Q}: Deviations from the straight line at the ends
  suggest potential outliers or leverage points. If the residuals are
  normally distributed, they should closely follow the reference line.
\item
  \textbf{Scale-Location}: Ideally, the points should be evenly spread
  across the range of fitted values, indicating equal variance.
\item
  \textbf{Residuals vs Leverage}: You are specifically looking for
  points outside the Cook's distance lines or with high leverage that
  could unduly influence the model.
\end{itemize}

\hypertarget{using-standardized-residuals}{%
\subsubsection{Using Standardized
Residuals}\label{using-standardized-residuals}}

For standardized residuals, you often look for residuals beyond the
(-2,2) range as potential outliers. You can customize your plots or add
additional code to specifically focus on standardized residuals:

\begin{Shaded}
\begin{Highlighting}[]
\CommentTok{\# Calculate standardized residuals}
\NormalTok{forward\_std\_res }\OtherTok{\textless{}{-}} \FunctionTok{rstandard}\NormalTok{(forward\_model)}
\NormalTok{backward\_std\_res }\OtherTok{\textless{}{-}} \FunctionTok{rstandard}\NormalTok{(backward\_model)}

\CommentTok{\# Plotting standardized residuals for Model 1}
\FunctionTok{plot}\NormalTok{(forward\_model}\SpecialCharTok{$}\NormalTok{fitted.values, forward\_std\_res, }\AttributeTok{ylim=}\FunctionTok{c}\NormalTok{(}\SpecialCharTok{{-}}\DecValTok{3}\NormalTok{,}\DecValTok{3}\NormalTok{), }\AttributeTok{xlab=}\StringTok{"Fitted Values"}\NormalTok{, }\AttributeTok{ylab=}\StringTok{"Standardized Residuals"}\NormalTok{)}
\FunctionTok{abline}\NormalTok{(}\AttributeTok{h=}\FunctionTok{c}\NormalTok{(}\SpecialCharTok{{-}}\DecValTok{2}\NormalTok{, }\DecValTok{2}\NormalTok{), }\AttributeTok{col=}\StringTok{"red"}\NormalTok{)}
\end{Highlighting}
\end{Shaded}

\includegraphics{Gadham_23783777_Atikant_24051868_files/figure-latex/unnamed-chunk-13-1.pdf}

\begin{Shaded}
\begin{Highlighting}[]
\CommentTok{\# Plotting standardized residuals for Model 2}
\FunctionTok{plot}\NormalTok{(backward\_model}\SpecialCharTok{$}\NormalTok{fitted.values, backward\_std\_res, }\AttributeTok{ylim=}\FunctionTok{c}\NormalTok{(}\SpecialCharTok{{-}}\DecValTok{3}\NormalTok{,}\DecValTok{3}\NormalTok{), }\AttributeTok{xlab=}\StringTok{"Fitted Values"}\NormalTok{, }\AttributeTok{ylab=}\StringTok{"Standardized Residuals"}\NormalTok{)}
\FunctionTok{abline}\NormalTok{(}\AttributeTok{h=}\FunctionTok{c}\NormalTok{(}\SpecialCharTok{{-}}\DecValTok{2}\NormalTok{, }\DecValTok{2}\NormalTok{), }\AttributeTok{col=}\StringTok{"red"}\NormalTok{)}
\end{Highlighting}
\end{Shaded}

\includegraphics{Gadham_23783777_Atikant_24051868_files/figure-latex/unnamed-chunk-13-2.pdf}

\hypertarget{comment-on-the-findings}{%
\subsubsection{Comment on the Findings}\label{comment-on-the-findings}}

The comment on the findings should include observations from the plots
regarding each model's compliance with linear regression assumptions.
Note any specific outliers, leverage points, or other anomalies that
might affect the model's validity or indicate a need for further model
refinement or data transformation.

\begin{enumerate}
\def\labelenumi{(\alph{enumi})}
\setcounter{enumi}{3}
\tightlist
\item
  (6 marks). Despite any inadequacies that you may or may not have
  identi ed above, you use the two models obtained in (b) to make
  predictions of Weight in the test set.
\item
  Produce a correctly drawn and labelled plot of predicted values
  against the ac- tual values in the test set, and obtain the root mean
  squared error of prediction (RMSEP) based on each fitted model.
\end{enumerate}

\begin{enumerate}
\def\labelenumi{(\roman{enumi})}
\setcounter{enumi}{1}
\tightlist
\item
  Using the RMSEPs and the plots you produced, comment on how well the
  models performed. Hint. Your answer must include a snippet of R code
  to calculate and plot the RMSEPs, the plots and comment. Answer: For
  part (d), we will use the two models developed from forward and
  backward selection methods to make predictions on the testing set and
  evaluate their performance using the root mean squared error of
  prediction (RMSEP). We will then visualize the predictions against
  actual values to understand the model's accuracy visually.
\end{enumerate}

\begin{Shaded}
\begin{Highlighting}[]
\DocumentationTok{\#\#\# R Code for Prediction and RMSEP Calculation}
\CommentTok{\# Load necessary library}
\FunctionTok{library}\NormalTok{(ggplot2)}

\CommentTok{\# Make predictions using both models}
\NormalTok{predictions\_forward }\OtherTok{\textless{}{-}} \FunctionTok{predict}\NormalTok{(forward\_model, }\AttributeTok{newdata =}\NormalTok{ testing)}
\NormalTok{predictions\_backward }\OtherTok{\textless{}{-}} \FunctionTok{predict}\NormalTok{(backward\_model, }\AttributeTok{newdata =}\NormalTok{ testing)}

\CommentTok{\# Calculate RMSEP for both models}
\NormalTok{rmsep\_forward }\OtherTok{\textless{}{-}} \FunctionTok{sqrt}\NormalTok{(}\FunctionTok{mean}\NormalTok{((testing}\SpecialCharTok{$}\NormalTok{Weight }\SpecialCharTok{{-}}\NormalTok{ predictions\_forward)}\SpecialCharTok{\^{}}\DecValTok{2}\NormalTok{))}
\NormalTok{rmsep\_backward }\OtherTok{\textless{}{-}} \FunctionTok{sqrt}\NormalTok{(}\FunctionTok{mean}\NormalTok{((testing}\SpecialCharTok{$}\NormalTok{Weight }\SpecialCharTok{{-}}\NormalTok{ predictions\_backward)}\SpecialCharTok{\^{}}\DecValTok{2}\NormalTok{))}

\CommentTok{\# Print RMSEP values}
\FunctionTok{print}\NormalTok{(}\FunctionTok{paste}\NormalTok{(}\StringTok{"RMSEP for Forward Selection Model:"}\NormalTok{, rmsep\_forward))}
\end{Highlighting}
\end{Shaded}

\begin{verbatim}
## [1] "RMSEP for Forward Selection Model: 12.9223506535226"
\end{verbatim}

\begin{Shaded}
\begin{Highlighting}[]
\FunctionTok{print}\NormalTok{(}\FunctionTok{paste}\NormalTok{(}\StringTok{"RMSEP for Backward Selection Model:"}\NormalTok{, rmsep\_backward))}
\end{Highlighting}
\end{Shaded}

\begin{verbatim}
## [1] "RMSEP for Backward Selection Model: 1.92851093325706"
\end{verbatim}

\begin{Shaded}
\begin{Highlighting}[]
\CommentTok{\# Plot of Predicted vs Actual values for both models}
\FunctionTok{ggplot}\NormalTok{() }\SpecialCharTok{+}
  \FunctionTok{geom\_point}\NormalTok{(}\FunctionTok{aes}\NormalTok{(}\AttributeTok{x =}\NormalTok{ testing}\SpecialCharTok{$}\NormalTok{Weight, }\AttributeTok{y =}\NormalTok{ predictions\_forward), }\AttributeTok{color =} \StringTok{"blue"}\NormalTok{) }\SpecialCharTok{+}
  \FunctionTok{geom\_point}\NormalTok{(}\FunctionTok{aes}\NormalTok{(}\AttributeTok{x =}\NormalTok{ testing}\SpecialCharTok{$}\NormalTok{Weight, }\AttributeTok{y =}\NormalTok{ predictions\_backward), }\AttributeTok{color =} \StringTok{"red"}\NormalTok{) }\SpecialCharTok{+}
  \FunctionTok{geom\_abline}\NormalTok{(}\AttributeTok{intercept =} \DecValTok{0}\NormalTok{, }\AttributeTok{slope =} \DecValTok{1}\NormalTok{, }\AttributeTok{linetype =} \StringTok{"dashed"}\NormalTok{, }\AttributeTok{color =} \StringTok{"black"}\NormalTok{) }\SpecialCharTok{+}
  \FunctionTok{labs}\NormalTok{(}\AttributeTok{title =} \StringTok{"Predicted vs Actual Weight"}\NormalTok{,}
       \AttributeTok{x =} \StringTok{"Actual Weight (kg)"}\NormalTok{,}
       \AttributeTok{y =} \StringTok{"Predicted Weight (kg)"}\NormalTok{) }\SpecialCharTok{+}
  \FunctionTok{scale\_color\_manual}\NormalTok{(}\AttributeTok{values =} \FunctionTok{c}\NormalTok{(}\StringTok{"blue"}\NormalTok{, }\StringTok{"red"}\NormalTok{), }\AttributeTok{labels =} \FunctionTok{c}\NormalTok{(}\StringTok{"Forward"}\NormalTok{, }\StringTok{"Backward"}\NormalTok{)) }\SpecialCharTok{+}
  \FunctionTok{theme\_minimal}\NormalTok{() }\SpecialCharTok{+}
  \FunctionTok{theme}\NormalTok{(}\AttributeTok{legend.title =} \FunctionTok{element\_blank}\NormalTok{())}
\end{Highlighting}
\end{Shaded}

\includegraphics{Gadham_23783777_Atikant_24051868_files/figure-latex/unnamed-chunk-14-1.pdf}

\hypertarget{explanation-and-output-interpretation-1}{%
\subsubsection{Explanation and Output
Interpretation}\label{explanation-and-output-interpretation-1}}

\begin{itemize}
\tightlist
\item
  \textbf{Predictions}: The \texttt{predict()} function is used to
  generate predictions from the models using the test set.
\item
  \textbf{RMSEP Calculation}: RMSEP is calculated by taking the square
  root of the mean squared error (MSE) between the actual and predicted
  weight values. It provides a measure of the average magnitude of the
  prediction errors.
\item
  \textbf{Plotting}: The plot displays the predicted values against the
  actual values for each model, with the ideal line (\texttt{y\ =\ x})
  added for reference. Points from the forward selection are in blue,
  and from the backward selection are in red.
\end{itemize}

\hypertarget{comment-on-model-performance}{%
\subsubsection{Comment on Model
Performance}\label{comment-on-model-performance}}

\begin{itemize}
\tightlist
\item
  \textbf{RMSEP Values}: Lower RMSEP values indicate better model
  performance. Compare the RMSEP for the two models to see which one
  predicts more accurately on the test set.
\item
  \textbf{Predicted vs Actual Plot}:

  \begin{itemize}
  \tightlist
  \item
    Ideal predictions would lie on the dashed line (\texttt{y\ =\ x}),
    indicating perfect agreement between predicted and actual values.
  \item
    Scatter around this line indicates the variability in predictions,
    with tighter clusters around the line suggesting better model
    performance.
  \item
    Observing whether one color (model) consistently lies closer to the
    line can help determine which model is more accurate.
  \end{itemize}
\end{itemize}

\hypertarget{comment}{%
\subsubsection{Comment}\label{comment}}

Based on the RMSEP values and the distribution of points in the plot,
you can provide a comprehensive comment on: - How well each model
predicts actual weights. - Which model appears to perform better and
why. - Any systematic errors observed in predictions (e.g.,
overpredictions or underpredictions at certain weight ranges). -
Implications for further model refinement or usage in practical
scenarios.

This process not only quantifies the performance of each model but also
visually inspects the adequacy and precision of the predictions,
offering a holistic view of how these models might perform in real-world
applications.

\end{document}
